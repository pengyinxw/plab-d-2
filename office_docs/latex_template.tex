\documentclass[10pt,letterpaper,final,conference]{IEEEtran}
\IEEEoverridecommandlockouts
\pagestyle{empty}
\usepackage{mathptmx}
\usepackage{times}
\usepackage[scaled=.92]{helvet}
\usepackage{ifpdf}
\ifpdf
\usepackage[pdftex]{graphicx}
\usepackage{epstopdf}
\else
\usepackage[dvips]{graphicx}
\fi
\usepackage{float}
\usepackage{cancel}
\usepackage{cite}
\newcommand{\Cut}[1]{}
\newcommand{\myind}[1]{\textit{\scriptsize #1}}
\newcommand{\tablesize}{\small}
\newcommand{\degree}{\ensuremath{^\circ\:}}              
\setlength{\parindent}{0.5cm}
\usepackage[ngerman]{babel}
%correctly encode umlauts
\usepackage[utf8]{inputenc}
\usepackage[T1]{fontenc}
\usepackage{lmodern}


% ============================================================================
% begin document
% ============================================================================

\begin{document}
\pagestyle{empty}

\title{Titel}

\author{\IEEEauthorblockN{Autor1, Autor2 und Autor3}
\vspace*{1em}
\IEEEauthorblockA{Leibniz Universität Hannover\\
Email: \{autor1, autor2, autor3\}\\@email.de
}
}


\renewcommand{\baselinestretch}{1}
% ============================================================================
% Abstract
% ============================================================================
\maketitle
\pagestyle{empty}
 \begin{abstract}
 \pagestyle{empty}
 \thispagestyle{empty}
Hier steht die Zusammenfassung.
\end{abstract}

% ============================================================================
% Introduction
% ============================================================================
\section{Einleitung}
Hier steht die Einleitung \cite{Cal2003}.

% ============================================================================
% Bibs
% ============================================================================
\bibliographystyle{IEEEtran}
\bibliography{latex_template}

\end{document}
